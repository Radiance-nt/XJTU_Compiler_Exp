\documentstyle[11pt,handout]{article}
%
% Copyright (c) 1995-1996 by Alex Aiken.  All rights reserved.
% Permission is granted to modify and distribute this document for
% for non-commercial purposes, so long as this copyright notice is retained
% in all copies.
%
% Side margins:
% Actual margin is 1 in + this number
\oddsidemargin -0.25in
\evensidemargin -0.25in

% Text width:
\textwidth 6.9in

% Top margin:
% Actual margin is 1.5 in + this number
\topmargin -.3in

% Text height:
\textheight 8.7in

% generally useful macros for writing
\newcommand{\TexDir}{/home3/aiken/tex}

% These allow switching interline spacing; the change takes effect immediately:

\makeatletter
\newcommand{\singlespacing}{\let\CS=\@currsize\renewcommand{\baselinestretch}{1}\tiny\CS}
\newcommand{\oneandahalfspacing}{\let\CS=\@currsize\renewcommand{\baselinestretch}{1.25}\tiny\CS}
\newcommand{\doublespacing}{\let\CS=\@currsize\renewcommand{\baselinestretch}{1.5}\tiny\CS}
%setspacingto sets the interline spacing to the value of its argument
% e.g., \setspacingto{1.5} is the same as \doublespacing
\newcommand{\setspacingto}[1]{\let\CS=\@currsize\renewcommand{\baselinestretch}{#1}\tiny\CS}
\makeatother

% nonumber
\newcommand{\nn}{\nonumber}

% Tab for hand-formatting:
\newcommand{\tab}{\hspace*{2em}}

% Angle brackets:
\newcommand{\la}{\langle}
\newcommand{\ra}{\rangle}

% s.t.
\newcommand{\st}{\mbox{\ s.t.\ }}

% otherwise
\newcommand{\ow}{\m{\rm otherwise}}

% if
\newcommand{\mif}{\m{\rm if\ }}

% Harpoons
\newcommand{\rh}{\rightharpoonup}
\newcommand{\lh}{\leftharpoonup}

% Denotational-semantics-style brackets and bottom:
\newcommand{\lbk}{\lbrack\!\lbrack}
\newcommand{\rbk}{\rbrack\!\rbrack}
\newcommand{\bottom}{\perp}

% Projection operator
\newcommand{\proj}{\!\downarrow\!}

% macros for mbox combined with another style
% (useful for changing typefaces in math mode)
\newcommand {\mboxbf}[1]{\mbox{{\bf #1}}}
\newcommand {\mboxit}[1]{\mbox{{\it #1}}}
\newcommand {\mboxem}[1]{\mbox{{\em #1}}}
\newcommand {\m}{\mbox}
\newcommand {\ch}{\rm}
	
% macro for creating a binary operator
%
% example:  \makebinop{\makebinop{\mybmod}{mod}
%	(duplicates the \bmod macro)
%
\def\makebinop#1#2{\def#1{\mskip-\medmuskip \mskip5mu
\mathbin{\rm #2} \penalty900 \mskip5mu \mskip-\medmuskip}}

% Environments for theorems, lemmas, etc.
\newtheorem{theorem}{Theorem}[section]
\newtheorem{lemma}[theorem]{Lemma}
\newtheorem{corollary}[theorem]{Corollary}
\newtheorem{definition}[theorem]{Definition}
\newtheorem{observation}[theorem]{Observation}
\newtheorem{fact}{Fact}[theorem]
\newtheorem{proposition}[theorem]{Proposition}
\newtheorem{example}[theorem]{Example}
\newtheorem{constraint}[theorem]{Constraint}
\newtheorem{axiom}[theorem]{Axiom}
\newtheorem{law}[theorem]{Law}
\newtheorem{algorithm}[theorem]{Algorithm}
\newtheorem{invariant}[theorem]{Invariant}

% Definition of the proof-environment:
\newenvironment{proof}{{\bf Proof:}\quad}{$\Box$}

% Definition of an eqnarray-like environment with an extra
% column for comments
\newcommand{\eeqnarray}[1]{\[\begin{array}{rcll} #1 \end{array}\]}

% Definition of a eqnarray-like environment for proofs of the
% form 
%     init     
%=>   step1   reason1
%=>   step2   reason2
%...
\newcommand{\peqnarray}[1]{\[\begin{array}{cll} #1 \end{array}\]}

%
% An environment for formatting programs. 
%
%	written by Hal Perkins
%	adapted by Charles Elkan	9/3/86
%	keyword macros by Anne Neirynck
%
% Usage :
%
% \begin{program}
% program text\\
% program text
% \end{program}
%
% The program environment is a tabbing environment with ten tab stops spaced
% evenly from the left of the page.  Initially the left margin is the second
% tab stop.  Use \+ to indent following lines one more tab stop, \- to undo
% the effect of \+, and \> at the beginning of a line to indent an extra tab.

\newlength{\pgmtab}          %  \pgmtab is the width of each tab in the
\setlength{\pgmtab}{2em}     %  program environment

% boxed program is like program, only boxed!
% This is useful for centering programs and preventing page breaks in programs.
% argument t or b is required.

\newenvironment{boxed-program}[1]{\begin{minipage}[#1]{9in}
\begin{tabbing}\hspace{\pgmtab}\=\hspace{\pgmtab}\=%
\hspace{\pgmtab}\=\hspace{\pgmtab}\=\hspace{\pgmtab}\=\hspace{\pgmtab}\=%
\hspace{\pgmtab}\=\hspace{\pgmtab}\=\hspace{\pgmtab}\=\hspace{\pgmtab}\=%
\hspace{\pgmtab}\=%
\kill}{\end{tabbing}\end{minipage}}

\newenvironment{program}{\begin{tabbing}\hspace{\pgmtab}\=\hspace{\pgmtab}\=%
\hspace{\pgmtab}\=\hspace{\pgmtab}\=\hspace{\pgmtab}\=\hspace{\pgmtab}\=%
\hspace{\pgmtab}\=\hspace{\pgmtab}\=\hspace{\pgmtab}\=\hspace{\pgmtab}\=%
\hspace{\pgmtab}\=%
\+\+\kill}{\end{tabbing}}

% The following commands should be used OUTSIDE math mode

\newcommand {\FUNCTION}{{\bf function\ }}
\newcommand {\BEGIN}{{\bf begin\ }}
\newcommand {\END}{{\bf end}}
\newcommand {\CASE}{{\bf case}}
\newcommand {\OF}{{\bf of}}
\newcommand {\SELECT}{{\bf select\ }}
\newcommand {\WHERE}{{\bf where\ }}
\newcommand {\DECLARE}{{\bf declare\ }}
\newcommand {\ARRAY}{{\bf array\ }}
\newcommand {\LET}{{\bf\ let\ }}
\newcommand {\IN}{{\bf\ in\ }}
\newcommand {\IF}{{\bf if\ }}
\newcommand {\THEN}{{\bf then\ }}
\newcommand {\ELSE}{{\bf else\ }}
\newcommand {\SKIP}{{\bf skip\ }}
\newcommand {\DO}{{\bf do\ }}
% OLD---don't use \OD
\newcommand {\OD}{{\bf od\ }}
\newcommand {\BY}{{\bf by\ }}
\newcommand {\LOOP}{{\bf loop\ }}
\newcommand {\WHILE}{{\bf while\ }}
\newcommand {\TO}{{\bf to\ }}
\newcommand {\DOWNTO}{{\bf down to\ }}
\newcommand {\FOR}{{\bf for\ }}
\newcommand {\FOREACH}{{\bf for each\ }}
\newcommand {\RETURN}{{\bf return\ }}
\newcommand {\REPEAT}{{\bf repeat\ }}
\newcommand {\UNTIL}{{\bf until\ }}
\newcommand {\LCOM}{$(\ast \;$}
\newcommand {\RCOM}{$\ast)$}
\newcommand {\GOTO}{{\bf goto\ }}

% The following commands are for use INSIDE math mode

\newcommand {\OP}[1]{\mbox{\sc #1}}
\newcommand {\op}[1]{\mbox{\sc #1}}
\newcommand {\mm}[1]{\mbox{\rm #1}\;}
\newcommand {\id}[1]{\mbox{\it #1\ }}

\newcommand {\ASSIGN}{\leftarrow }
\newcommand {\MIN}{\OP{min} }
\newcommand {\MOD}{\; {\bf{\rm mod}} \; } 
\newcommand {\LAMBDA}{{\bf \lambda\ }}
\newcommand {\FALSE}{{\em FALSE\ }}

% Miscellaneous notation

\newcommand {\And}{\wedge}
\newcommand {\Or}{\vee}

\newcommand {\thus}{{\dot{. \: .}\;}}

\newcommand {\bigO}[1]{{\cal O}(#1)}

\newcommand {\app}{\!\!:\!}
\newcommand {\hastype}{::}
\newcommand{\hast}{:}
\newcommand{\qt}[1]{\mbox{``#1''}}
\newcommand{\TexComment}[1]{}

\newcommand{\dq}{\m{\tt "}}
\newcommand{\flatqt}[1]{\m{\dq #1 \dq}}
\newcommand{\seq}{\subseteq}
\newcommand{\derives}{\vdash}

% for writing grammars
\newcommand{\grammar}{::=}
\newcommand{\gor}{\,|\,}

% macros for writing inference rules
\newcommand{\infrule}[2]{\displaystyle{\displaystyle\strut{#1}} \over %
                                        {\displaystyle\strut {#2}}}
\newcommand{\cinfrule}[3]{\parbox{14cm}{\hfil$\infrule{#1}{#2}$\hfil}\parbox{4cm}{$\,#3$\hfil}}

\begin{document}

\handout{9}{6}{Programming Assignment IV \\
Due Tuesday, November 7}

% Three macros for defining:
% 	Unix elements: filenames and program (sans serif)
%	Sather164 elements: literal tokens (typewriter)
%	C elements: function and variable names (boldface)
%
\def\U#1{{\sf{}#1}}
\def\S#1{{\tt{}#1}} % NB: we often use \verb+...+ for this also
\def\C#1{{\bf{}#1}}


\section{Introduction} 

In this assignment you will implement the static semantics of Cool.
You will use the abstract syntax trees (AST) built by
your parser from assignment 3 to check that a program is in conformance
with the Cool specification. Your static semantic
component should reject erroneous programs; for correct programs, it
must gather certain information for use by the code generator.

This assignment has much more room for design
decisions than previous assignments. Your program is correct if it checks
programs against the specification. There is no ``right'' way to do
the assignment, but there are wrong ways. There are a number of
standard practices which we think make life easier and we will try to
convey them to you. However, what you do is largely up to you.
Whatever you decide to do, be prepared to justify your solution.

You will need to refer to the typing rules, identifier scoping rules,
and other restrictions of Cool as defined in the
CoolAid.  You will also need to add methods and data members to the 
AST class definitions for this phase.  The functions
the tree package provides are documented in the handout for programming
assignment 3 and in the header files \U{cool-tree.h} and \U{tree.h}.
You can modify the class definitions by changing either
\U{cool-tree.h} or \U{cool-tree.handcode.h} (the latter is {\tt \#included}
by the former).

\section{Files and Directories}

To get the assignment type
\begin{verbatim}
make -f ~cs164/assignments/PA4/Makefile
\end{verbatim}
in an appropriate directory.  This command copies a number
of files to your directory, some of them with read-only permission.
As usual, you should not modify files that are read-only.
Please read and follow the directions in the \U{README} file.

The files that you will need to modify are:
\begin{itemize}

\item \U{semant.cc}
This file contains a start on a semantic analysis phase, written in C++.
Put the code for your semantic analysis phase in this
file.  The skeleton only includes things required to
correctly meet the interface with the code generator.  (This interface
is discussed in detail below.)  Very little of real interest is
included.  Unlike previous assignments, this skeleton doesn't even compile.

\item \U{semant.h}
This file is the header file for \U{semant.cc}.

\item \U{cool-tree.handcode.h}, \U{cool-tree.h}
These files are where user-defined extensions to the abstract syntax
tree nodes are placed.  You can modify either file, but you will probably
find it easiest to modify \U{cool-tree.h}.  
You may add new \C{\#define} statements,
but do not modify the existing declarations, except for the \C{{\em class}\_EXTRA}
macros.  You may add any fields you wish to the \C{{\em class}\_EXTRA} macros.

\item \U{symtab.h} \\
This file contains code for a simple symbol table module.
You are not required to modify these files, but you are free to do so if the symbol
table manager does not meet your needs.  
For example, you may wish to change the \C{SymtabEntry} template in
\U{symtab.h} to add information to the symbol table.
Document any changes you make to the original code.

\item \U{good.cl} \U{bad.cl} \\
These files test a few semantic features.
You should add tests to ensure that \U{good.cl} exercises as many legal
semantic combinations as possible and that \U{bad.cl} exercises as many
kinds of semantic errors as possible.  It is not possible to exercise
all possible combinations in one file; you are only responsible for
achieving reasonable coverage.  Explain your tests in these files and put
any overall comments in the \U{README} file.

\item \U{README} \\
This file will contain the write-up for your assignment.  For this
assignment it is critical that you explain design decisions, how your
code is structured, and why you believe that the design is a good one
(i.e., why it leads to a correct and robust program).  It is part of
the assignment to explain things in text as well as to comment your
code.  Inadequate README files will be penalized more heavily in this
assignment, as the README is the major guideline we have to
understanding your code.

Make sure that the name, student ID, and login of each
group member is in the \U{README} file.
\end{itemize}

As usual, there are other files used in the assignment that are symbolically
linked to your directory or are included from \U{~cs164/include/PA4}.
You should not modify these files.
Almost all of these files have have been described in previous assignments.
The exceptions are \U{ast.flex} and\U{ast.y}, which implement
lexical analysis and parsing for a textual representation of Cool ASTs.
Recall that there are two versions of \U{coolc}: a normal executable
and a shell script that glues separate parser, semantic analysis, and
code generation phases together via pipes.  Each phase uses \U{dumptype}
to print the AST to the pipe, and the next phase uses the AST parser
to reconstruct the tree data structure.\footnote{One may wonder: Why
have two versions of \U{coolc}?  The ``phased'' version of
\U{coolc} greatly simplifies the structure of the programming assignments by
removing the need to guarantee that your code for one phase link with all 
components for other phases of the course compiler.}

All software supplied with this assignment is supported on both
the HP and DEC machines.  Remember to run \U{gmake clean} if you switch
architectures. 

\section{Testing the Semantic Analyzer}

You will need a working scanner and parser to test your semantic analyzer.
There is a script \U{mysemant} that combines a scanner, parser,
and your semantic analyzer.
See the \U{README} file for instructions on how to use either your own
components or the components from \U{coolc}.  It is wise to test your
semantic analyzer with the \U{coolc} scanner and parser at least once,
because we will grade your semantic analyzer using \U{coolc}'s scanner
and parser.

Once you are confident that your semantic analyzer is working, you
should test \U{mycoolc} on both good and bad inputs to see if everything 
is working.
Remember, bugs in the semantic analyzer may manifest themselves in the
code generated or only when the compiled program is executed under
\U{spim}.

The \U{-p} and \U{-l} flags are available for debugging information
from the lexer and the parser.  These flags are unlikely to be of much
use in the semantic analyzer.  We have provided you with a \U{-s} flag
for debugging you semantic analyzer; see the \U{README} file for instructions
on how to use this flag.

\section{Tree Traversal}

As a result of assignment 3, your parser builds abstract syntax trees.
The file \U{dumptype.cc} illustrates how to traverse the AST and gather
information from it. This algorithmic style---a recursive traversal of
a complex tree structure---is very important, because it is a very natural
way to structure many computations on ASTs.

Your programming task for this assignment is to
1) traverse the tree, 2) manage various pieces of information that you
glean from the tree, and 3) use that information to enforce the
semantics of Cool.  One traversal of the AST is called a ``pass''.
You must make at least two passes (and probably more) over the
AST to check everything.

There is a place in the AST for you to attach customized information. In
\U{cool-tree.handcode.h} there is are \C{{\em class}\_EXTRA} macros.
The AST constructors contain the members listed in the \C{{\em class}\_EXTRA}
macro for each AST node of type {\em class}.  You may add members to the macros
to make nodes of that type hold whatever information you wish. You may 
also modify \U{cool-tree.h} directly.

\section{Inheritance}

Inheritance relationships specify a directed graph of class
dependencies.  A typical requirement of most languages with
inheritance is that the inheritance graph be acyclic.
It is up to your semantic checker to enforce this requirement. One
fairly easy way to do this is to construct a representation of the
type graph and then check for cycles.

In addition, Cool has restrictions on inheriting from the basic classes
(see the manual).  It is also an error if class \U{A}
inherits from class \U{B} but class \U{B} is not defined.

The skeleton \U{semant.cc} includes appropriate definitions of all the
basic classes. You will need to incorporate these classes into the
inheritance hierarchy.

We suggest that you divide your semantic analysis phase into two
smaller components.  First, check that the inheritance graph is
well-defined, meaning that all the restrictions on inheritance are
satisfied.  If the inheritance graph is not well-defined, it is
acceptable to abort compilation (after printing appropriate error
messages, of course!).  Second, check all the other semantic
conditions.  It is much easier to implement this second component if
one knows that the inheritance graph is legal.


\section{Naming and Scoping}

A major portion of any semantic checker is the management of names.
The specific problem is determining which declaration is in effect for
each use of an identifier, especially when names can be reused. For
example, if \C{i} is declared in two \U{let} expressions, one nested within the
other, then wherever \C{i} is referenced  the semantics of the language
specify which declaration is in effect.  It is the job of the semantic
checker to keep track of which declaration a name refers to.

As discussed in class, a {\em symbol table} is usually used to manage names.
We provide a module to manage symbols and you
are free to do with it as you like. Our module allows you to enter, exit
and augment scopes as needed.

Besides the identifier \C{self}, which is implicitly bound in every class,
there are four ways that an object name can be introduced in Cool:
\begin{itemize}
\item attribute definitions

\item formal parameters of methods

\item let expressions

\item branches of case statements
\end{itemize}

In addition to object names, there are also method names and class names.
It is, of course, an error to use any name that has no matching declaration.

Remember that neither classes, methods, nor attributes need be
declared before use.  Think about how this affects your analysis.


\section{Type Checking}

Type checking is another major function of the semantic analyzer.
The semantic analyzer must check that valid types are declared where
required.  For example, the return types of methods must be declared.
Using this information, the semantic analyzer must also verify that every
expression has a valid type according to the type rules.  The type rules
are discussed in detail in the CoolAid and the course lecture notes.

One difficult issue is what to do if an expression doesn't have a
valid type according to the rules.  First, an error message should be
printed with the line number and a description of what went wrong.  It
is relatively easy to give informative error messages in the semantic
analysis phase, because it is generally obvious what the error is.  We
expect you to give informative error messages.  Second, the semantic analyzer
should attempt to recover and continue.  A good semantic analyzer
will avoid cascading errors using any of several standard techniques.
We do expect your semantic analyzer to recover, but we do not expect
it to avoid cascading errors.  A simple recovery mechanism is to assign 
the type \U{Object} to any expression that cannot otherwise be given a 
type (we used this method in \U{coolc}).

\section{Code Generator Interface}

For the semantic analyzer to work correctly with the rest of the \U{coolc}
compiler, some care must be taken to adhere to the interface with the
code generator.  We have deliberately adopted a very simple, naive interface to
avoid cramping your creative impulses in semantic analysis.  However, there
is one thing you must do.
For every expression node, its \U{type} field (defined in \U{cool-tree.handcode.h})
must be set to the \U{Symbol} naming the type inferred by your type
checker.  This
\U{Symbol} must be the result of the \U{add\_string} method of the
\U{idtable}.  The special expression \U{no\_expr} must be assigned the
type \U{No\_type}.

\section{Output and Grading}

For incorrect programs, the output of semantic analysis is error
messages.  You are expected to recover from all errors except for
ill-formed class hierarchies.  You are also expected to produce
complete and informative errors.  Assuming the inheritance hierarchy
is well-formed, the semantic checker should catch and report all
semantic errors in the program.  When in doubt, use
\C{coolc} as a guide in determining what informative error messages
should say.  Your error messages need not be identical to those of
\C{coolc}.

We have supplied you with a simple error reporting method
\C{semant\_error}.  This routine takes a filename and the AST node where
the error occurred, and it returns the error stream after it has
printed an error header.  The filename should be the file in which the
error occurs.  The driver in \U{semanttest.c} sets the global
\C{curr\_filename} to the file in which parsing is taking place; this
name is used when creating \U{Class\_} nodes to record the filename.
Thus, the \U{Class\_} nodes store the file in which the class was
defined (recall that class definitions cannot be split across files).
In an error message, the line number of the error message is obtained
from the AST node where the error is detected and the file name is
obtained from the enclosing class.

For correct programs, the output is a type-annotated abstract syntax
tree.  You will be graded on whether your semantic phase correctly
annotates ASTs with types and on whether your semantic phase works
correctly with the {\tt coolc} code generator.

You are also expected to program in good, structured style.  You
should spend some time thinking about the class definitions you will
use.

\section{Remarks}

The semantic analysis phase is by far the largest component of the compiler
so far.  Our solution is approximately 1300 lines of well-documented C++.
You will find the assignment easier if you take some time to design the
semantic checker prior to coding.  Ask yourself:
\begin{itemize}
\item What requirements do I need to check?
\item When do I need to check a requirement?
\item When is the information needed to check a requirement generated?
\item Where is the information I need to check a requirement?
\end{itemize}
If you can answer these questions for an aspect of Cool, implementing
a solution should be straightforward.
At a high level, your semantic checker will have to perform the
following major tasks:
\begin{enumerate}
\item Gather all classes.
\item Build an inheritance graph.
\item Check that the graph is well-formed.
\item For each class
\begin{enumerate}
\item Traverse the AST, gathering all visible declarations in a symbol table.
\item Check each expression for type correctness.
\end{enumerate}
\end{enumerate}
This list of tasks is not exhaustive; it is up to you to faithfully implement
the specification in the manual.

\end{document}

