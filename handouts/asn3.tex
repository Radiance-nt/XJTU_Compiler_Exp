\documentstyle[11pt,handout]{article}
%
% Copyright (c) 1995-1996 by Alex Aiken.  All rights reserved.
% Permission is granted to modify and distribute this document for
% for non-commercial purposes, so long as this copyright notice is retained
% in all copies.
%
% Side margins:
% Actual margin is 1 in + this number
\oddsidemargin -0.25in
\evensidemargin -0.25in

% Text width:
\textwidth 6.9in

% Top margin:
% Actual margin is 1.5 in + this number
\topmargin -.3in

% Text height:
\textheight 8.7in

% generally useful macros for writing
\newcommand{\TexDir}{/home3/aiken/tex}

% These allow switching interline spacing; the change takes effect immediately:

\makeatletter
\newcommand{\singlespacing}{\let\CS=\@currsize\renewcommand{\baselinestretch}{1}\tiny\CS}
\newcommand{\oneandahalfspacing}{\let\CS=\@currsize\renewcommand{\baselinestretch}{1.25}\tiny\CS}
\newcommand{\doublespacing}{\let\CS=\@currsize\renewcommand{\baselinestretch}{1.5}\tiny\CS}
%setspacingto sets the interline spacing to the value of its argument
% e.g., \setspacingto{1.5} is the same as \doublespacing
\newcommand{\setspacingto}[1]{\let\CS=\@currsize\renewcommand{\baselinestretch}{#1}\tiny\CS}
\makeatother

% nonumber
\newcommand{\nn}{\nonumber}

% Tab for hand-formatting:
\newcommand{\tab}{\hspace*{2em}}

% Angle brackets:
\newcommand{\la}{\langle}
\newcommand{\ra}{\rangle}

% s.t.
\newcommand{\st}{\mbox{\ s.t.\ }}

% otherwise
\newcommand{\ow}{\m{\rm otherwise}}

% if
\newcommand{\mif}{\m{\rm if\ }}

% Harpoons
\newcommand{\rh}{\rightharpoonup}
\newcommand{\lh}{\leftharpoonup}

% Denotational-semantics-style brackets and bottom:
\newcommand{\lbk}{\lbrack\!\lbrack}
\newcommand{\rbk}{\rbrack\!\rbrack}
\newcommand{\bottom}{\perp}

% Projection operator
\newcommand{\proj}{\!\downarrow\!}

% macros for mbox combined with another style
% (useful for changing typefaces in math mode)
\newcommand {\mboxbf}[1]{\mbox{{\bf #1}}}
\newcommand {\mboxit}[1]{\mbox{{\it #1}}}
\newcommand {\mboxem}[1]{\mbox{{\em #1}}}
\newcommand {\m}{\mbox}
\newcommand {\ch}{\rm}
	
% macro for creating a binary operator
%
% example:  \makebinop{\makebinop{\mybmod}{mod}
%	(duplicates the \bmod macro)
%
\def\makebinop#1#2{\def#1{\mskip-\medmuskip \mskip5mu
\mathbin{\rm #2} \penalty900 \mskip5mu \mskip-\medmuskip}}

% Environments for theorems, lemmas, etc.
\newtheorem{theorem}{Theorem}[section]
\newtheorem{lemma}[theorem]{Lemma}
\newtheorem{corollary}[theorem]{Corollary}
\newtheorem{definition}[theorem]{Definition}
\newtheorem{observation}[theorem]{Observation}
\newtheorem{fact}{Fact}[theorem]
\newtheorem{proposition}[theorem]{Proposition}
\newtheorem{example}[theorem]{Example}
\newtheorem{constraint}[theorem]{Constraint}
\newtheorem{axiom}[theorem]{Axiom}
\newtheorem{law}[theorem]{Law}
\newtheorem{algorithm}[theorem]{Algorithm}
\newtheorem{invariant}[theorem]{Invariant}

% Definition of the proof-environment:
\newenvironment{proof}{{\bf Proof:}\quad}{$\Box$}

% Definition of an eqnarray-like environment with an extra
% column for comments
\newcommand{\eeqnarray}[1]{\[\begin{array}{rcll} #1 \end{array}\]}

% Definition of a eqnarray-like environment for proofs of the
% form 
%     init     
%=>   step1   reason1
%=>   step2   reason2
%...
\newcommand{\peqnarray}[1]{\[\begin{array}{cll} #1 \end{array}\]}

%
% An environment for formatting programs. 
%
%	written by Hal Perkins
%	adapted by Charles Elkan	9/3/86
%	keyword macros by Anne Neirynck
%
% Usage :
%
% \begin{program}
% program text\\
% program text
% \end{program}
%
% The program environment is a tabbing environment with ten tab stops spaced
% evenly from the left of the page.  Initially the left margin is the second
% tab stop.  Use \+ to indent following lines one more tab stop, \- to undo
% the effect of \+, and \> at the beginning of a line to indent an extra tab.

\newlength{\pgmtab}          %  \pgmtab is the width of each tab in the
\setlength{\pgmtab}{2em}     %  program environment

% boxed program is like program, only boxed!
% This is useful for centering programs and preventing page breaks in programs.
% argument t or b is required.

\newenvironment{boxed-program}[1]{\begin{minipage}[#1]{9in}
\begin{tabbing}\hspace{\pgmtab}\=\hspace{\pgmtab}\=%
\hspace{\pgmtab}\=\hspace{\pgmtab}\=\hspace{\pgmtab}\=\hspace{\pgmtab}\=%
\hspace{\pgmtab}\=\hspace{\pgmtab}\=\hspace{\pgmtab}\=\hspace{\pgmtab}\=%
\hspace{\pgmtab}\=%
\kill}{\end{tabbing}\end{minipage}}

\newenvironment{program}{\begin{tabbing}\hspace{\pgmtab}\=\hspace{\pgmtab}\=%
\hspace{\pgmtab}\=\hspace{\pgmtab}\=\hspace{\pgmtab}\=\hspace{\pgmtab}\=%
\hspace{\pgmtab}\=\hspace{\pgmtab}\=\hspace{\pgmtab}\=\hspace{\pgmtab}\=%
\hspace{\pgmtab}\=%
\+\+\kill}{\end{tabbing}}

% The following commands should be used OUTSIDE math mode

\newcommand {\FUNCTION}{{\bf function\ }}
\newcommand {\BEGIN}{{\bf begin\ }}
\newcommand {\END}{{\bf end}}
\newcommand {\CASE}{{\bf case}}
\newcommand {\OF}{{\bf of}}
\newcommand {\SELECT}{{\bf select\ }}
\newcommand {\WHERE}{{\bf where\ }}
\newcommand {\DECLARE}{{\bf declare\ }}
\newcommand {\ARRAY}{{\bf array\ }}
\newcommand {\LET}{{\bf\ let\ }}
\newcommand {\IN}{{\bf\ in\ }}
\newcommand {\IF}{{\bf if\ }}
\newcommand {\THEN}{{\bf then\ }}
\newcommand {\ELSE}{{\bf else\ }}
\newcommand {\SKIP}{{\bf skip\ }}
\newcommand {\DO}{{\bf do\ }}
% OLD---don't use \OD
\newcommand {\OD}{{\bf od\ }}
\newcommand {\BY}{{\bf by\ }}
\newcommand {\LOOP}{{\bf loop\ }}
\newcommand {\WHILE}{{\bf while\ }}
\newcommand {\TO}{{\bf to\ }}
\newcommand {\DOWNTO}{{\bf down to\ }}
\newcommand {\FOR}{{\bf for\ }}
\newcommand {\FOREACH}{{\bf for each\ }}
\newcommand {\RETURN}{{\bf return\ }}
\newcommand {\REPEAT}{{\bf repeat\ }}
\newcommand {\UNTIL}{{\bf until\ }}
\newcommand {\LCOM}{$(\ast \;$}
\newcommand {\RCOM}{$\ast)$}
\newcommand {\GOTO}{{\bf goto\ }}

% The following commands are for use INSIDE math mode

\newcommand {\OP}[1]{\mbox{\sc #1}}
\newcommand {\op}[1]{\mbox{\sc #1}}
\newcommand {\mm}[1]{\mbox{\rm #1}\;}
\newcommand {\id}[1]{\mbox{\it #1\ }}

\newcommand {\ASSIGN}{\leftarrow }
\newcommand {\MIN}{\OP{min} }
\newcommand {\MOD}{\; {\bf{\rm mod}} \; } 
\newcommand {\LAMBDA}{{\bf \lambda\ }}
\newcommand {\FALSE}{{\em FALSE\ }}

% Miscellaneous notation

\newcommand {\And}{\wedge}
\newcommand {\Or}{\vee}

\newcommand {\thus}{{\dot{. \: .}\;}}

\newcommand {\bigO}[1]{{\cal O}(#1)}

\newcommand {\app}{\!\!:\!}
\newcommand {\hastype}{::}
\newcommand{\hast}{:}
\newcommand{\qt}[1]{\mbox{``#1''}}
\newcommand{\TexComment}[1]{}

\newcommand{\dq}{\m{\tt "}}
\newcommand{\flatqt}[1]{\m{\dq #1 \dq}}
\newcommand{\seq}{\subseteq}
\newcommand{\derives}{\vdash}

% for writing grammars
\newcommand{\grammar}{::=}
\newcommand{\gor}{\,|\,}

% macros for writing inference rules
\newcommand{\infrule}[2]{\displaystyle{\displaystyle\strut{#1}} \over %
                                        {\displaystyle\strut {#2}}}
\newcommand{\cinfrule}[3]{\parbox{14cm}{\hfil$\infrule{#1}{#2}$\hfil}\parbox{4cm}{$\,#3$\hfil}}

\begin{document}
\handout{6}{9}{Programming Assignment III \\ 
Due Thursday, October 12}

% Three macros for defining:
% 	Unix elements: filenames and program (sans serif)
%	Cool elements: literal tokens (typewriter)
%	C elements: function and variable names (boldface)
%
\def\U#1{{\sf{}#1}}
\def\S#1{{\tt{}#1}} % NB: we often use \verb+...+ for this also
\def\C#1{{\bf{}#1}}

\section{Introduction}

In this assignment you will write a parser for Cool.  The assignment
makes use of two tools: the parser generator \U{bison} and a package
for generating and manipulating trees.  You will not use use the
generating part of the package, but only the manipulators of a
(already generated) tree definition we will give you. The output of
your parser will be an abstract syntax tree (AST).  The AST will be
constructed using \U{bison}'s semantic actions.

You certainly will need to refer to the syntactic structure of Cool,
found in Figure 1 of the CoolAid, as well as other portions of the
reference manual.  There is a chapter on \U{bison} in the course
reader.  There is also a section in the textbook on \U{yacc}, which is
a close relative of \U{bison}.  The tree package is described in the
{\em Tour} handout.  You will need the tree package information for this and
future assignments.

There is a lot of information in this handout and you need to know most of
it to write a working parser.  Please read the handout thoroughly.

\section {Files and Directories}

To get the assignment type
\begin{verbatim}
gmake -f ~cs164/assignments/PA3/Makefile
\end{verbatim}
in a directory named \U{PA3}.  This command copies a number
of files to your directory, some of them with read-only permission.
As usual, you should not modify files that are read-only.  If you
insist on making your own versions of these files, you are on your own.
If your code doesn't compile with the original files (including the original
\U{Makefile}) you will receive little credit, because your program will fail
all of the automatically run test cases.
Please read and follow the directions in the \U{README} file.

The files that you need to modify are:
\begin{itemize}

\item \U{cool.y} \\
This file contains a start towards a parser description for Cool.
You will need to add more rules.
The section before the first \S{\%\%} is mostly complete;
all you need to do is add \S{\%type} declarations for new nonterminals.
(We have given you names and \S{\%type} declarations for the terminals.)
The rule section is very incomplete.

\item \U{good.cl} and \U{bad.cl} \\
These test a few features of the grammar.  You should add tests to
ensure that \U{good.cl} exercises every legal construction of the
grammar and that \U{bad.cl} exercises as many types of parsing errors
as possible in a single file.  Explain your tests in these files and
put any overall comments in the \U{README} file.

\item \U{README} \\
As usual, 
this file will contain the write-up for your assignment.
Explain your design decisions, your test cases, and why you believe your
program is correct and robust.
It is part of the assignment to explain things in text as well as to
comment your code.
Make sure that the name, student ID, and login of each
group member is in the \U{README} file.
\end{itemize}%


The files that your code will link with and include, but that you should
not modify, are:
\begin{itemize}
\item \U{parsetest.cc} \\
This file contains a simple driver for the parser.

\item \U{cool-tree.aps} \\
This file is a specification of the Cool tree language written in
a special language.  It defines a type system and describes the kinds
of nodes used for the abstract syntax trees.

\item \U{cool.h}, \U{cool-tree.h}, \U{cool-tree.handcode.h}, and \U{tree.h} \\
The header file \U{cool.h} defines basic types used by the compiler.
The other files define the routines that let you create and examine
abstract syntax trees and lists of trees.
It may interest you that the second file was automatically generated from
\U{cool-tree.aps}.

\item \U{dumptype.cc} \\
This file implements a routine for printing out ASTs.  The algorithm used
illustrates the  use of virtual functions to recursively traverse of an AST; you will need to write algorithms in a similar style in future assignments.

\item \U{handle\_files.cc} \\
This file implements a simple mechanism for parsing multiple input files.

\item \U{parsetest.a} \\
This is the object code for \U{handle\_flags.cc} and \U{handle\_files.cc}.

\item \U{lexer.a} \\
This is the object code for the \U{coolc} lexical analyzer.

\item \U{mylexer.a} \\
This is an alternative to \U{lexer.a}.  Use this library if you
want to run the parser with your own lexical analyzer (see \U{README}
for details).
 
\end{itemize}%

{\bf Important:} All software supplied with this assignment is supported on both
the HP and DEC machines.  Remember to run \U{gmake clean} if you switch
architectures.  To make sure your code will link correctly on both architectures,
you should place \U{~cs164/bin} at the beginning of your \U{PATH} variable.

\section{Testing the Parser}

You will need a working scanner to test the parser.  You may use either
your own scanner or the \U{coolc} scanner.  See the \U{README}
file for instructions on how to build a working parser either way.
Don't automatically assume that the scanner (whichever one you use!) is bug
free---latent bugs in the scanner may cause mysterious problems in the
parser.  

Note that \U{parsetest} has a \U{-p} flag for debugging
the parser; using this flag causes lots of information about what the
parser is doing to be printed on \S{stdout}.  In addition, the \U{-l}
flag is available for generating debugging output from the
scanner.  Using these two flags together is sometimes useful for
isolating strange behavior in the parser.

Once you are confident that your parser is working, try \U{make mycoolc} to
build a complete compiler that includes your parser.  You should
test this compiler on both good and bad inputs to see if everything is working.
Remember, bugs in your parser may manifest themselves anywhere.

Your parser will be graded using our lexical analyzer.  Thus,
even if you do most of the work using your own scanner you should test
your parser with the \U{coolc} scanner before turning in the assignment.
As always, type \S{make turnin} to turn in your assignment; see the
\U{README} file.

\section{Parser Output}

Your semantic actions should build an AST for the input.  The root
(and only the root) of the AST should be of type \S{Program}. The
semantic action for the start symbol assigns the root of the AST
to a global variable \S{ast\_root} (also of type \S{Program}, of course).
For programs that parse successfully, the output of \S{parsetest} is
a listing of the AST.

For programs that have errors, the output is the error messages of the
parser.  We have supplied you with a function \S{yyerror} in 
\U{cool.y} that prints error messages in a standard format; please 
do not modify this function.  You should not invoke \S{yyerror} directly
in the semantic actions; \S{bison} automatically invokes \S{yyerror} 
when it detects a problem.

Your parser need only work for programs contained in a single file---don't
worry about compiling multiple files.

\section{Error Handling}

You should use \U{bison}'s \S{error} pseudo-nonterminal to add error
handling capabilities in the parser.  The purpose of \S{error} is to
permit the parser to continue after some anticipated error.  It is not
a panacea and the parser may become completely confused.  See the
\U{bison} documentation for how best to use \S{error}.  In your
\U{README}, describe which errors you attempt to catch.  Your test
file \U{bad.cl} should have some instances that illustrate the errors
from which your parser can recover.  To receive full credit, you
parser should recover in at least the following situations:
\begin{itemize}

\item If there is an error in a class definition but the class is
terminated properly and the next class is syntactically correct,
 the parser should be able to restart at the next class definition.

\item Similarly, the parser should recover from errors in features
(going on to the next feature), a {\tt let} binding (going on to the next
variable), and an expression inside a {\tt begin-end} block.

\end{itemize}

Do not be overly concerned about the the line numbers that appear in
the error messages your parser generates.  If your parser is working
correctly, the line number will generally be the line where the error
occurred.  For erroneous constructs broken across multiple lines, the
line number will probably be the last line of the construct.

\section{The Tree Package}
\label{sec-tree}

There is an extensive discussion of the APS tree package for Cool abstract
syntax trees in the {\em Tour}.  You will need most of that information
to write a working parser.

\section {Remarks}

You may use \U{bison}'s precedence declarations, but only for expressions.
Do not use precedence declarations blindly---do not respond
to a shift-reduce conflict in your grammar by adding precedence
rules until it goes away.  If you find yourself making up rules for
things other than operators in expressions,
you are probably doing something wrong.

You must declare \S{bison} ``types'' for your non-terminals and terminals that
have attributes.  For example, in the skeleton \U{cool.y}
is the declaration:
\begin{verbatim}
%type <class_> class
\end{verbatim}
This declaration says that the non-terminal \S{class} has type \S{<class\_>}.
The use of the word ``type'' is misleading here; what it really means is that
the attribute for the non-terminal \S{class} is stored in the \S{class\_}
member of the \S{union} declaration in \U{cool.y}, which has type
\S{Class\_}.   By specifying the type  
\begin{verbatim}
%type <member_name> X Y Z ...
\end{verbatim}
you instruct \C{bison} that the attributes of non-terminals (or terminals)
{\tt X}, {\tt Y}, and {\tt Z} have a type appropriate for the member
{\tt member\_name} of the union.

All the union members and their types have similar names by
design.  It is a coincidence in the example above that the non-terminal \S{class}
has the same name as a union member.

It is critical that you declare the correct types for 
the attributes of grammar symbols; failure to do so virtually guarantees
that your parser won't work.  You do not need to declare types for
symbols of your grammar that do not have attributes.

The \U{g++} type checker complains if you use the tree constructors with the
wrong type parameters.  If you ignore the warnings, your program may
crash when the constructor notices that it is being used incorrectly.
Moreover, \U{bison} may complain if you make type errors.
Heed any warnings.  Don't be surprised if your program crashes when
\U{bison} or \U{g++} give warning messages.

\end{document}

